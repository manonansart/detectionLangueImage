Ce PAO nous a permis d'approfondir nos connaissances dans le domaine du traitement d'image notamment avec l'outil d'OCR qu'est Ocropus et de voir les principes théoriques appliqués lors du traitement de l'image mais aussi comment CLD2 reconnait une langue à partir d'un texte.\\

De plus, ce projet fut un bon moyen de mélanger diverses technologies telles que le php ainsi que le java orienté Android.Nous avons pu voir que faire communiquer des systèmes avec des technologies de développement différentes n'est pas forcément quelque chose de facile. Malgré cela, nous sommes parvenus à un résultat très proche de ce que l'on souhaitait faire au départ.\\

Pour ce qui est des améliorations à apporter au projet par la suite s'il est repris, nous avons pensé à ceci :
\begin{itemize}
 \item Tout d'abord, il faudrait améliorer l'interface de l'application avec l'ajout d'un message d'avancement du traitement ainsi qu'une modification de la chaîne de caractère du résultat de CLD2 afin de n'avoir que la langue et le pourcentage.
 \item Ensuite, il serait peut être possible avec plus de connaissances en python d'essayer de diminuer le temps de traitement par Ocropus en modifiant le code. Ou bien en utilisant d'autres bibliothèques d'OCR en remplacement d'Ocropus.
 \item Le développement ou l'utilisation d'un explorateur de fichier sur android afin de transférer au serveur de traitement une photo de document directement présente sur le mobile.
 \item Enfin, il sera utile de rendre l'application disponible en dehors du réseau de l'INSA afin de pouvoir traiter une image de n'importe où.
\end{itemize}

Pour finir, nous avons beaucoup appris de ce PAO qui a été très instructif pour notre formation d'ingénieur. En effet, nous avons découvert de nouveaux outils non étudiés dans le cadre de notre scolarité tels que la programmation Android et vu des applications de nos cours théorique comme le traitement d'image par Ocropus. 