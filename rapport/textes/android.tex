\subsection{Conception}

Nous avons développé cette application pour les systèmes d'exploitation Android. La programmation pour Android est basée sur le langage java que nous avons vu l'année dernière en cours de programmation avancée. Néanmoins, même si nous possédions de bonnes bases java, il nous a fallu plusieurs semaines afin de savoir développer sur Android car celui-ci fonctionne différemment du langage java que nous avions pu voir. \\

Une fois le langage assimilé, nous nous sommes attaqué à la conception de l'application. Nous avons tout d'abord élaboré un dessin papier des différentes vus de notre application. Au fur et à mesure que nous avancions dans le developpement nous nous sommes rendu compte qu'il serait plus compliqué que prévu de réaliser toutes ces vus. Nous avons donc opté pour une application plus simple mais qui réppond à nos attentes.\\

la figure ci dessous présente le menu de démarrage de l'application : 
	\begin{figure}[H]
		\centering
		\includegraphics[scale=0.1]{images/appliMenu.png}
		\caption{Menu principal}
		\label{fig:image}
	\end{figure}

\subsection{Fonctionnement}
L'intégralité du code de l'application est disponible dans l'archive fournis.\\

Lorsque nous sommes dans le menu principal il nous suffit d'appuyer sur le bouton "prendre une photo" afin que l'application fasse appelle à l'application appareil photo de notre mobile. Pour faire ceci, nous utilisons un listener qui lorsque l'on clic sur le bouton l'applcation crée un intent afin d'utiliser l'application en charge des photos. Ceci nous à permis de ne pas redévelopper l'appareil photo déjà présent.\\

Une fois que nous avons pris la photo, il nous suffit de valider et nous retournons à l'accueil. L'application envoie alors grâce à notre classe DataSender l'image par protocole Http sur l'adresse du script php de notre serveur. Une fois cela fait, l'application reste à l'écoute du serveur afin de récupérer la chaine de caractère renvoyée. Cette chaine contient le résultat du traitement par Ocropus puis CLD2.\\

Nous pouvons alors reprendre une photo ou bien quitter l'application.

\subsection{Problèmes rencontrés}

Tout d'abord, l'apprentissage du java pour Android est apparu plus difficile que prévu car nous ne pensions pas qu'il y avait autant de différences. De plus nous avons eu quelques soucis pour installer le SDK android sur Eclipse. C'est pourquoi nous avons pris un peu de retard au début de ce projet.\\

Ensuite, il a été assez compliqué de faire communiquer l'application avec le serveur. Plus précisément, nous ne savions pas quel type de protocole était le plus adapté afin d'envoyer une photo sur un serveur. Après avoir consulté Mr Nicolas Malandin, nous avons donc opté pour le protocole Http.\\

Finalement, nous avons toujours des problèmes de réponse du serveur, ce qui fait que par moment nous ne sommes pas certain de recevoir une réponse. En effet, le traitement de l'image par Ocropus prend plusieurs minutes suivant la taille du texte à traiter.