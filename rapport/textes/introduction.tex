Le but initial de ce projet était de travailler avec la librairie Olena d'Epita afin de détecter des photos dans des images de documents. Après le téléchargement et la compilation de la librairie, nous avons tenté de la tester et de rechercher les fonctions utiles pour l'application que nous voulions en faire. Nous avons découvert que la librairie était bien plus complexe que ce que nous pensions, et qu'elle ne fournissait pas de fonction permettant de trouver directement une photo dans une image. Nous avons donc décidé de choisir un autre sujet, toujours dans le cadre de la reconnaissance d'image de document. Pendant ce Projet d'Approfondissement et d'Ouverture, nous allons utiliser la bibliothèque ocropus pour reconnaître les caractères à partir d'une image de documents puis en détecter la langue grâce au module de détection de langue cld2 développé par google.